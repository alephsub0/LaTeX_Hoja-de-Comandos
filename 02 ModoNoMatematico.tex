\documentclass[a4,11pt]{aleph-notas}

% -- Paquetes adicionales
\usepackage{enumitem}
\usepackage{multicol}
\usepackage{booktabs}
\usepackage{aleph-comandos}

% -- Datos del libro
\universidad{Proyecto Alephsub0}
\autor{Andrés Merino}
\materia{\LaTeX}
\nota{Comandos básicos}
\tema{}
\fecha{Mayo de 2021}


% -- Logo y colores
\logouno[4.5cm]{LogoAlephsub0-02}
\longtitulo{0.6\linewidth}

% -- Otros comandos
\geometry{margin=10mm}
\setlength{\parskip}{0ex}
\usepackage{ragged2e}
    \setlength{\RaggedRightParindent}{1em}

\usepackage{url}
\usepackage{listings}
\lstset{%
    basicstyle=\scriptsize\ttfamily,  % the size of the fonts 
    columns=fixed, % anything else is horrifying
    showspaces=false,       % show spaces using underscores?
    showstringspaces=false, % underline spaces within strings?
    showtabs=false,% show tabs within strings?
    xleftmargin=1.5em,      % left margin space
}
\lstdefinestyle{inline}{basicstyle=\ttfamily}

\usepackage{titlesec}
    \titleformat{\section}[runin]{\color{colortext}\bf}{}{0em}{}
    \titlespacing*{\section}{0em}{0.65ex}{0.67em}

\pagestyle{empty}

\newlength{\MyLen}

\begin{document}
\encabezado

\thispagestyle{empty}
\RaggedRight

\begin{multicols}{2}

\section{Clases de documento}
Utilizar al inicio del documento:
\verb!\documentclass{!\textit{clase}\verb!}!. Utilice \verb!\begin{document}! para iniciar el contenido y \verb!\end{document}! para tenrminarlo.
\begin{center}
\begin{tabular}{@{}ll@{}}
\verb!book!    & Para formato de libro. \\
\verb!report!  & Para formato de reporte. \\
\verb!article! & Para formato de artículo. \\
\verb!letter!  & Para formato de carta. \\
\verb!beamer!  & Para formato de presentación.
\end{tabular}
\end{center}

\section{Opciones básicas de \texttt{documentclass}}
Utilizar\linebreak \verb!\documentclass[!\textit{opt,opt}\verb!]{!\textit{class}\verb!}!.
\begin{center}
\begin{tabular}{@{}ll@{}}
\texttt{10pt}/\texttt{11pt}/\texttt{12pt} & Tamaño de fuente. \\
\texttt{a4paper}/\texttt{a5paper} & Tamaño del papel.
\end{tabular}
\end{center}

\section{Paquetes básicos}

Utilizar antes de \verb!\begin{document}!. 
Modo de uso: \verb!\usepackage[!\textit{opc}\verb!]{!\textit{paquete}\verb!}!

\begin{center}
\begin{tabular}{@{}ll@{}}
    \texttt{inputenc}  &  Reconoce caracteres latinos (\texttt{utf8}). \\
    \texttt{fontenc}   &  Despliega caracteres latinos (\texttt{T1}). \\
    \texttt{babel}   &  Cambia idioma a español (\texttt{spanish}).     \\
    \texttt{multicol}  & Permite utilizar varias columnas.\\
    \texttt{graphicx}  & Permite importar imágenes.\\
    \texttt{url}       & Permite insertar direcciones electrónicas.\\
    \texttt{geometry}  & Permite cambiar la geometría de la página.
\end{tabular}
\end{center}

\section{Medidas y geometría}
Las unidades de medida permitidas son:
\begin{center}
\begin{tabular}{@{}llll@{}}
    \verb!mm! & milímetros 
    &\verb!\textwidth! & ancho de texto\\
    \verb!cm! & centímetros 
    &\verb!\linewidth! & ancho de linea\\
    \verb!in! & pulgadas 
    &\verb!px! & pixeles 
\end{tabular}
\end{center}
Utilizar el comando \verb!\geometry!, por ejemplo
\verb!\geometry{margin=2cm}!


\section{Titulo}
Estos comandos van antes de \verb!\begin{document}!. El comando \verb!\maketitle! va en la parte superior del documento.

\begin{center}
\begin{tabular}{@{}ll@{}}
    \verb!\author{!\textit{texot}\verb!}! & Autor del documento. \\
    \verb!\title{!\textit{texto}\verb!}!  & Título del  documento. \\
    \verb!\date{!\textit{texto}\verb!}!   & Fecha. \\
\end{tabular}
\end{center}


\section{Estructura del documento}
Utilizar un \texttt{*}, como en \verb!\section*{!\textit{título}\verb!}!, para no numerar.

\begin{multicols}{2}
\verb!\part{!\textit{título}\verb!}!  \\
\verb!\chapter{!\textit{título}\verb!}!  \\
\verb!\section{!\textit{título}\verb!}!  \\
\verb!\subsection{!\textit{título}\verb!}!  \\
\verb!\subsubsection{!\textit{título}\verb!}!  \\
\verb!\paragraph{!\textit{título}\verb!}!  \\
\verb!\subparagraph{!\textit{título}\verb!}!
\end{multicols}



\section{Referencias}

\begin{center}
\settowidth{\MyLen}{\texttt{.pageref.marker..}}
\begin{tabular}{@{}p{\the\MyLen}%
                @{}p{\linewidth-\the\MyLen}@{}}
    \verb!\label{!\textit{etiq}\verb!}!   
        &  Agrega una etiqueta. \\
    \verb!\ref{!\textit{etiq}\verb!}!   
        &  Regresa el número de la etiqueta \\
    \verb!\pageref{!\textit{etiq}\verb!}! 
        &  Regresa el número de página de la etiqueta. \\
    \verb!\footnote{!\textit{texto}\verb!}!  
        &  Genera una nota al pie. \\
\end{tabular}
\end{center}

\section{Flotantes}
El \textit{lugar} puede ser \texttt{t}=arriba,
\texttt{h}=aquí, \texttt{b}=abajo, \texttt{p}=página aparte, \texttt{H}=obligatoriamente aquí (paquete \texttt{float}. Las leyendas y etiquetas deben estar dentro del ambiente.

\begin{center}
\settowidth{\MyLen}{\texttt{.begin.equationlugar.}}
\begin{tabular}{@{}p{\the\MyLen}%
                @{}p{\linewidth-\the\MyLen}@{}}
    \verb!\begin{table}[!\textit{lugar}\verb!]!
        &  Agrega un cuadro numerado. \\
    \verb!\begin{figure}[!\textit{lugar}\verb!]!
        &  Agrega una figura numerada. \\
    \verb!\caption{!\textit{texto}\verb!}!
        &  Coloca la leyenda. \\
\end{tabular}
\end{center}


\section{Tipo de fuente}

\begin{center}
\begin{tabular}{@{}lll@{}}
\textit{Comando} & \textit{Declaración} & \textit{Efecto} \\
    \verb!\textrm{!\textit{texto}\verb!}!            
    & \verb!{\rmfamily !\textit{texto}\verb!}!       
    & \textrm{Roman family} 
    \\
    \verb!\textsf{!\textit{texto}\verb!}!            
    & \verb!{\sffamily !\textit{texto}\verb!}!       
    & \textsf{Sans serif family} 
    \\
    \verb!\texttt{!\textit{texto}\verb!}!            
    & \verb!{\ttfamily !\textit{texto}\verb!}!       
    & \texttt{Typewriter family} 
    \\
    \verb!\textbf{!\textit{texto}\verb!}!            
    & \verb!{\bfseries !\textit{texto}\verb!}!       
    & \textbf{Bold series} 
    \\
    \verb!\textit{!\textit{texto}\verb!}!            
    & \verb!{\itshape !\textit{texto}\verb!}!       
    & \textit{Italic shape} 
    \\
    \verb!\textsl{!\textit{texto}\verb!}!            
    & \verb!{\slshape !\textit{texto}\verb!}!       
    & \textsl{Slanted shape} 
    \\
    \verb!\textsc{!\textit{texto}\verb!}!            
    & \verb!{\scshape !\textit{texto}\verb!}!       
    & \textsc{Small Caps shape} 
    \\
    \verb!\emph{!\textit{texto}\verb!}!              
    & \verb!{\em !\textit{texto}\verb!}!       
    & \emph{Emphasized} 
\end{tabular}
\end{center}


\section{Tamaño de fuente}
Estas declaraciones pueden utilizarse de la forma
\verb!{\small! \ldots\verb!}!, o sin las llaves para afectar a todo el documento.

\begin{center}
\begin{tabular}{@{}llll@{}}
    \verb!\tiny!
    &\tiny{tiny} 
    &\verb!\large!
    &\large{large}
    \\
    \verb!\scriptsize!
    &\scriptsize{scriptsize}
    &\verb!\Large!
    &\Large{Large}
    \\
    \verb!\footnotesize!
    &\footnotesize{footnotesize} 
    &\verb!\LARGE!
    &\LARGE{LARGE} 
    \\
    \verb!\small!
    &\small{small} 
    &\verb!\huge!
    &\huge{huge} 
    \\
    \verb!\normalsize!
    &\normalsize{normalsize} 
    & \verb!\Huge!
    &\Huge{Huge}
\end{tabular}
\end{center}



\section{Listas}
\begin{center}
\begin{tabular}{@{}ll@{}}
    \verb!\begin{enumerate}!   &  Lista numerada. \\
    \verb!\begin{itemize}!     &  Lista no numerada. \\
    \verb!\begin{description}! &  Lista de descripción. \\
    \verb!\item! \textit{texto}&  Agregar ítem. \\
    \verb!\item[!\textit{x}\verb!]! \textit{texto} &  Utiliza \textit{x} en lugar de la viñeta. \\
\end{tabular}
\end{center}

Ejemplos:\\
\begin{minipage}{0.45\linewidth}
\begin{enumerate}
    \item 
        Primero
    \item 
        Segundo
\end{enumerate}
\end{minipage}
%
\begin{minipage}{0.45\linewidth}
\begin{verbatim}
\begin{enumerate}
    \item 
        Primero
    \item 
        Segundo
\end{enumerate}
\end{verbatim}
\end{minipage}




\section{Justificación}

\begin{center}
\begin{tabular}{@{}ll@{}}
\textit{Ambiente}  &  \textit{Declaración}  \\
\verb!\begin{center}!      & \verb!\centering!  \\
\verb!\begin{flushleft}!  & \verb!\raggedright! \\
\verb!\begin{flushright}! & \verb!\raggedleft!  \\
\end{tabular}
\end{center}


\section{Símbolos}

\begin{center}
\begin{tabular}{@{}ll@{\hspace{3em}}ll@{\hspace{3em}}ll@{}}
    \& & \verb!\&! 
    &
    \_ & \verb!\_! 
    &
    \ldots & \verb!\ldots! 
       \\
    \$ & \verb!\$! 
    &
    \^{} & \verb!\^{}! 
    &
    \textbar 
    &  \verb!\textbar! 
    \\
    \% &  \verb!\%! 
    &
    \~{} &  \verb!\~{}! 
    &
    \textbullet & \verb!\textbullet!   
    \\
    \# &  \verb!\#! 
    &
    \S &  \verb!\S! 
    &
    \textbackslash & \verb!\textbackslash!
\end{tabular}
\end{center}

\section{Varios}

\begin{center}
\settowidth{\MyLen}{\texttt{.rule.w..h...} }
\begin{tabular}{@{}p{\the\MyLen}%
                @{}p{0.95\linewidth-\the\MyLen}@{}}
    \verb!``comillas''!  
        &  ``comillas''. \\
    \verb!\today!  
        &  \today. \\
    \verb!\hspace{!$l$\verb!}! 
        & Espacio horizontal. \\
    \verb!\vspace{!$l$\verb!}! 
        & Espacio vertical. \\
    \verb!\rule{!$w$\verb!}{!$h$\verb!}! 
        & Linea de ancho $w$ y altura $h$. \\
\end{tabular}
\end{center}


\section{Tablas}
Utilizar el ambiente \verb!\begin{tabular}{!\textit{cols}\verb!}!, donde \textit{col} es la definición de las columnas, puede tomar las siguientes especificaciones:
\begin{center}


\begin{tabular}{@{}ll@{}}
    \texttt{l}    
        &   Columna justificada a la izquierda.  \\
    \texttt{c}    
        &   Columna centrada.  \\
    \texttt{r}    
        &   Columna justificada a la derecha. \\
    \verb!p{!\textit{ancho}\verb!}!  
        &  Columna de ancho fijo. \\ 
    \verb!|!      
        &  Colocar línea vertical entre columnas. 
\end{tabular}
\end{center}

Además, con el comando \verb!\hline!  se colocan lineas horizontales. Con el paquete \texttt{booktabs} se tiene las siguientes lineas horizontales.

\begin{center}
\begin{tabular}{@{}ll@{}}
    \verb!\toprule! & Linea horizontal de inicio de la tabla.  \\
    \verb!\midrule! & Linea horizontal de mitad de la tabla.  \\
    \verb!\bottomrule! & Linea horizontal de final de la tabla.  \\
\end{tabular}
\end{center}

\begin{minipage}{0.45\linewidth}
\begin{tabular}{cc}
    \toprule
    Número & Precio \\
    \midrule
    1 & \$ 2.00\\
    2 & \$ 2.00\\
    \bottomrule
\end{tabular}
\end{minipage}
%
\begin{minipage}{0.45\linewidth}
\begin{verbatim}
\begin{tabular}{cc}
    \toprule
    Número & Precio \\
    \midrule
    1 & \$ 2.00\\
    2 & \$ 2.00\\
    \bottomrule
\end{tabular}
\end{verbatim}
\end{minipage}


\section{Gráficos}
Para la inclusión de gráficos es necesario el paquete \texttt{graphicx}, en caso de ser archivos vectoriales se requiere que el compilador sea \texttt{LaTeX}, de ser mapas de bits se requiere el compilador \texttt{pdfLaTeX}. El comando para incluir la imagen es
\verb!\includegraphics[!\textit{ops}\verb!]{!\textit{archivo}\verb!}!, por ejemplo

\noindent
\verb!\includegraphics[width=0.9\linewidth]{grafico1}!

\section{Leer más} 
Learn \LaTeX{} in 30 minutes en \url{https://es.overleaf.com/learn/latex/Learn_LaTeX_in_30_minutes}.


\end{multicols}
\end{document}
