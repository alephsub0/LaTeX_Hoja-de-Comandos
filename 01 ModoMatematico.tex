\documentclass[a4,11pt]{aleph-notas}

% -- Paquetes adicionales
\usepackage{enumitem}
\usepackage{multicol}
\usepackage{amsmath, amsthm, amssymb}
\usepackage{aleph-comandos}

% -- Datos del libro
\universidad{Proyecto Alephsub0}
\autor{Andrés Merino}
\materia{\LaTeX}
\nota{Comandos matemáticos básicos}
\tema{}
\fecha{Mayo de 2021}


% -- Logo y colores
\logouno[4.5cm]{LogoAlephsub0-02}
\longtitulo{0.6\linewidth}

 

% -- Otros comandos
\geometry{margin=10mm}
\setlength{\parskip}{0ex}
\usepackage{ragged2e}
    \setlength{\RaggedRightParindent}{1em}

\usepackage{url}
\usepackage{listings}
\lstset{%
    basicstyle=\scriptsize\ttfamily,  % the size of the fonts 
    columns=fixed, % anything else is horrifying
    showspaces=false,       % show spaces using underscores?
    showstringspaces=false, % underline spaces within strings?
    showtabs=false,% show tabs within strings?
    xleftmargin=1.5em,      % left margin space
}
\lstdefinestyle{inline}{basicstyle=\ttfamily}

\usepackage{titlesec}
    \titleformat{\section}[runin]{\color{colortext}\bf}{}{0em}{}
    \titlespacing*{\section}{0em}{0.65ex}{0.67em}

\pagestyle{empty}


\begin{document}
\encabezado

\thispagestyle{empty}
\RaggedRight

\begin{multicols}{2}

\section{Primera regla}
Todos los caracteres matemáticos deben ir en ambiente matemático. Así, para ``el valor de $x$ is $7$'' ingresar `\lstinline[style=inline]!el valor de $x$ es $7$!' o `\lstinline[style=inline]!el valor de \(x\) es \(7\)!'.

\section{Plantilla}
El documento debe contener al menos esto.
\begin{lstlisting}
\documentclass{article}

\usepackage[utf8]{inputenc}
\usepackage[T1]{fontenc}
\usepackage[spanish]{babel}
\usepackage{amsmath, amssymb}

\begin{document}
  --cuerpo del documento--
\end{document}     
\end{lstlisting}


\section{Estructuras comunes}
\begin{center}
  \begin{tabular}{@{} llll @{}}
    $x^2$ & \lstinline[style=inline]!x^2!
     & 
    $\sqrt{2}$, $\sqrt[n]{3}$ &\lstinline[style=inline]!\sqrt{2}!,
        \lstinline[style=inline]!\sqrt[n]{3}! 
\\     
    $x_{i,j}$ &\lstinline[style=inline]!x_{i,j}!
    & 
    $\frac{2}{3}$, $2/3$ & \lstinline[style=inline]!\frac{2}{3}!,
    \lstinline[style=inline]!2/3!
  \end{tabular}
\end{center}

\section{Tipos de letra}
Usar \lstinline[style=inline]!$\mathcal{A}$! para:
\begin{center}
    $ \mathcal{A}\mathcal{B}\mathcal{C} 
      \mathcal{D}\mathcal{E}\mathcal{F} 
      \mathcal{G}\mathcal{H}\mathcal{I} 
      \mathcal{J}\mathcal{K}\mathcal{L} 
      \mathcal{M}\mathcal{N}\mathcal{O} 
      \mathcal{P}\mathcal{Q}\mathcal{R} 
      \mathcal{S}\mathcal{T}\mathcal{U} 
      \mathcal{V}\mathcal{W}\mathcal{X} 
      \mathcal{Y}\mathcal{Z}  $
\end{center}
Usar \lstinline[style=inline]!$\mathbb{R}$! para:
\begin{center}
    $ \mathbb{A}\mathbb{B}\mathbb{C} 
      \mathbb{D}\mathbb{E}\mathbb{F} 
      \mathbb{G}\mathbb{H}\mathbb{I} 
      \mathbb{J}\mathbb{K}\mathbb{L} 
      \mathbb{M}\mathbb{N}\mathbb{O} 
      \mathbb{P}\mathbb{Q}\mathbb{R} 
      \mathbb{S}\mathbb{T}\mathbb{U} 
      \mathbb{V}\mathbb{W}\mathbb{X} 
      \mathbb{Y}\mathbb{Z}  $
\end{center}

\section{Letras griegas}
\begin{center}
\begin{tabular}{@{}llll@{}}
    $\alpha$&\lstinline[style=inline]!\alpha! 
    &$\xi$, $\Xi$&\lstinline[style=inline]!\xi!,\lstinline[style=inline]!\Xi!    \\
    $\beta$&\lstinline[style=inline]!\beta! 
    &o&\lstinline[style=inline]!o!  \\
    $\gamma$, $\Gamma$&\lstinline[style=inline]!\gamma!,\lstinline[style=inline]!\Gamma!
    &$\pi$, $\Pi$&\lstinline[style=inline]!\pi!, \lstinline[style=inline]!\Pi!\\
    $\delta$, $\Delta$&\lstinline[style=inline]!\delta!,\lstinline[style=inline]!\Delta! 
    &$\varpi$&\lstinline[style=inline]!\varpi! \\ 
    $\epsilon$&\lstinline[style=inline]!\epsilon! 
    &$\rho$&\lstinline[style=inline]!\rho! \\
    $\varepsilon$&\lstinline[style=inline]!\varepsilon! 
    &$\varrho$&\lstinline[style=inline]!\varrho! \\
    $\zeta$&\lstinline[style=inline]!\zeta! 
    &$\sigma$, $\Sigma$&\lstinline[style=inline]!\sigma!,\lstinline[style=inline]!\Sigma! \\
    $\eta$&\lstinline[style=inline]!\eta! 
    &$\varsigma$&\lstinline[style=inline]!\varsigma!\\
    $\theta$, $\Theta$&\lstinline[style=inline]!\theta!,\lstinline[style=inline]!\Theta! 
    &$\tau$& \lstinline[style=inline]!\tau!   \\
    $\vartheta$&\lstinline[style=inline]!\vartheta!    
    &$\upsilon$, $\Upsilon$&\lstinline[style=inline]!\upsilon!,\lstinline[style=inline]!\Upsilon!   \\
    $\iota$&\lstinline[style=inline]!\iota!  
    &$\phi$, $\Phi$&\lstinline[style=inline]!\phi!, \lstinline[style=inline]!\Phi!  \\
    $\kappa$&\lstinline[style=inline]!\kappa!
    &$\varphi$&\lstinline[style=inline]!\varphi!\\
    $\lambda$, $\Lambda$&\lstinline[style=inline]!\lambda!, \lstinline[style=inline]!\Lambda! 
    &$\chi$&\lstinline[style=inline]!\chi! \\
    $\mu$&\lstinline[style=inline]!\mu!    
    &$\psi$, $\Psi$&\lstinline[style=inline]!\psi!, \lstinline[style=inline]!\Psi!\\
    $\nu$&\lstinline[style=inline]!\nu!  
    &$\omega$, $\Omega$&\lstinline[style=inline]!\omega!, \lstinline[style=inline]!\Omega! 
\end{tabular}
\end{center}

\columnbreak
\section{Lógica y conjuntos}
\begin{center} 
\begin{tabular}{@{}*{6}{l@{\hspace{1em}}}@{}}
    $\cup$& \lstinline[style=inline]!\cup!
    &$\forall$& \lstinline[style=inline]!\forall! 
    &$\equiv$& \lstinline[style=inline]!\equiv!   \\
    $\cap$& \lstinline[style=inline]!\cap!
    &$\exists$& \lstinline[style=inline]!\exists!
    & $\aleph$& \lstinline[style=inline]!\aleph!   \\  
    $\subseteq$& \lstinline[style=inline]!\subseteq!
    &$\neg$& \lstinline[style=inline]!\neg!
    &$\iff$& \lstinline[style=inline]!\iff!\\
    $\in$& \lstinline[style=inline]!\in!   
    &$\land$& \lstinline[style=inline]!\land!
    &$\smallsetminus$& \lstinline[style=inline]!\smallsetminus!\\
    $\notin$& \lstinline[style=inline]!\notin!     
    &$\lor$& \lstinline[style=inline]!\lor!
    &$\Rightarrow$& \lstinline[style=inline]!\Rightarrow!\\
    $\not\in$& \lstinline[style=inline]!\not\in!  
    &$\vdash$& \lstinline[style=inline]!\vdash!
    &$\nRightarrow$& \lstinline[style=inline]!\nRightarrow! \\ 
    $\varnothing$& \lstinline[style=inline]!\varnothing! 
    &$\models$& \lstinline[style=inline]!\models!
    &$\circ$& \lstinline[style=inline]!\circ!\\
\end{tabular}
\end{center}
La negación de un operador, como $\not\subseteq$, se obtiene con \lstinline[style=inline]!\not\subseteq!.

\section{Decoraciones}
\begin{center}
\begin{tabular}{@{}*{6}{l@{\hspace{1em}}}@{}}
    $f'$&\lstinline[style=inline]!f'!
    &$\dot{a}$&\lstinline[style=inline]!\dot{a}!      
    &$\tilde{x}$&\lstinline[style=inline]!\tilde{x}!  \\
    $f''$&\lstinline[style=inline]!f''!
    &$\ddot{a}$&\lstinline[style=inline]!\ddot{a}!
    &$\bar{x}$&\lstinline[style=inline]!\bar{x}!\\
    $f^{*}$&\lstinline[style=inline]!f^{*}!
    &$\hat{x}$&\lstinline[style=inline]!\hat{x}!
    &$\vec{x}$&\lstinline[style=inline]!\vec{x}!
\end{tabular}
\end{center}
Para decorar más de un símbolo se puede utilizara
\begin{center}
\begin{tabular}{@{}*{2}{l@{\hspace{1em}}}@{}}
    $\overline{x+y}$&\lstinline[style=inline]!\overline{x+y}!\\
    $\widehat{x+y}$&\lstinline[style=inline]!\widehat{x+y}!\\
    $\overrightarrow{AB}$&\lstinline[style=inline]!\overrightarrow{AB}!\\
    $\underbrace{x+y}_{|A|}$&\lstinline[style=inline]!\underbrace{x+y}_{|A|}!
\end{tabular}
\end{center}

\section{Puntos}
Utilizar puntos bajos en las listas 
$\{0,1,2,\,\ldots\}$, ingresado como \lstinline[style=inline]!\{0,1,2,\,\ldots\}!. 

Utilizar puntos medios en las sumas o productos,
$1+\cdots+100$, ingresado como \lstinline[style=inline]!1+\cdots+100!.

Se pueden colocar también puntos verticales $\vdots$, \lstinline[style=inline]!\vdots!, y diagonales $\ddots$, \lstinline[style=inline]!\ddots!.

\section{Nombres romanos} Ingresar \lstinline[style=inline]!\tan(x)!, ($\tan(x)$)
con barra invertida, en lugar de \lstinline[style=inline]!tan(x)!, ($tan(x)$).
\begin{center}
  \begin{tabular}{@{}*{6}{l@{\hspace{1.5em}}}@{}}
    $\sen$& \lstinline[style=inline]!\sen!
    &$\senh$& \lstinline[style=inline]!\senh!     
    &$\arcsen$& \lstinline[style=inline]!\arcsen! \\    
    $\cos$& \lstinline[style=inline]!\cos!
    &$\cosh$& \lstinline[style=inline]!\cosh!     
    &$\arccos$& \lstinline[style=inline]!\arccos! \\   
    $\tan$& \lstinline[style=inline]!\tan!
    &$\tanh$& \lstinline[style=inline]!\tanh!      
    &$\arctan$& \lstinline[style=inline]!\arctan! \\
    $\sec$& \lstinline[style=inline]!\sec!
    &$\coth$& \lstinline[style=inline]!\coth!     
    &$\min$& \lstinline[style=inline]!\min!  \\
    $\csc$& \lstinline[style=inline]!\csc!  
    &$\det$& \lstinline[style=inline]!\det!
    &$\max$& \lstinline[style=inline]!\max!  \\
    $\cot$& \lstinline[style=inline]!\cot!
    &$\dim$& \lstinline[style=inline]!\dim!
    &$\inf$& \lstinline[style=inline]!\inf!\\
    $\exp$& \lstinline[style=inline]!\exp! 
    &$\ker$& \lstinline[style=inline]!\ker!      
    &$\sup$& \lstinline[style=inline]!\sup!\\
    $\log$& \lstinline[style=inline]!\log!
    &$\deg$& \lstinline[style=inline]!\deg!
    &$\liminf$& \lstinline[style=inline]!\liminf!\\
    $\ln$& \lstinline[style=inline]!\ln!
    &$\arg$& \lstinline[style=inline]!\arg!
    &$\limsup$& \lstinline[style=inline]!\limsup!\\
    $\lg$& \lstinline[style=inline]!\lg!
    &$\mcd$& \lstinline[style=inline]!\mcd!  
    &$\lim$& \lstinline[style=inline]!\lim! \\
    % &$\hom$& \lstinline[style=inline]!\hom!   \\
  \end{tabular}
\end{center}

\columnbreak
\section{Otros símbolos}
\begin{center}
\begin{tabular}{@{}*{6}{l@{\hspace{1.2em}}}@{}}
    $<$& \lstinline[style=inline]!<!  
    &$\angle$& \lstinline[style=inline]!\angle!      
    &$\cdot$& \lstinline[style=inline]!\cdot!\\
    $\leq$& \lstinline[style=inline]!\leq!
    &$\perp$& \lstinline[style=inline]!\perp!
    &$\pm$& \lstinline[style=inline]!\pm!\\
    $>$& \lstinline[style=inline]!>!  
    &$\ell$& \lstinline[style=inline]!\ell!  
    &$\mp$& \lstinline[style=inline]!\mp!   \\
    $\geq$& \lstinline[style=inline]!\geq!
    &$\parallel$& \lstinline[style=inline]!\parallel!      
    &$\times$& \lstinline[style=inline]!\times!   \\
    $\neq$& \lstinline[style=inline]!\neq!
    &$45^{\circ}$& \lstinline[style=inline]!45^{\circ}!
    &$\div$& \lstinline[style=inline]!\div!   \\
    $\ll$& \lstinline[style=inline]!\ll!        
    &$\cong$& \lstinline[style=inline]!\cong!        
    &$\ast$& \lstinline[style=inline]!\ast!   \\
    $\gg$& \lstinline[style=inline]!\gg!      
    &$\ncong$& \lstinline[style=inline]!\ncong!        
    &$\mid$& \lstinline[style=inline]!\mid!   \\
    $\approx$& \lstinline[style=inline]!\approx!      
    &$\sim$& \lstinline[style=inline]!\sim!      
    &$\nmid$& \lstinline[style=inline]!\nmid! \\
    $\asymp$& \lstinline[style=inline]!\asymp!      
    &$\simeq$& \lstinline[style=inline]!\simeq!        
    &$n!$& \lstinline[style=inline]+n!+    \\
    $\equiv$& \lstinline[style=inline]!\equiv!      
    &$\nsim$& \lstinline[style=inline]!\nsim!      
    &$\partial$& \lstinline[style=inline]!\partial! \\
    $\prec$& \lstinline[style=inline]!\prec!        
    &$\oplus$& \lstinline[style=inline]!\oplus!
    &$\nabla$& \lstinline[style=inline]!\nabla!    \\
    $\preceq$& \lstinline[style=inline]!\preceq!      
    &$\ominus$& \lstinline[style=inline]!\ominus!   
    &$\hbar$& \lstinline[style=inline]!\hbar!    \\
    $\succ$& \lstinline[style=inline]!\succ!        
    &$\odot$& \lstinline[style=inline]!\odot!      
    &$\circ$& \lstinline[style=inline]!\circ!\\
    $\succeq$& \lstinline[style=inline]!\succeq!      
    &$\otimes$& \lstinline[style=inline]!\otimes!  
    &$\star$& \lstinline[style=inline]!\star!   \\
    $\propto$& \lstinline[style=inline]!\propto!      
    &$\oslash$& \lstinline[style=inline]!\oslash!    
    &$\surd$& \lstinline[style=inline]!\surd!   \\
    $\doteq$& \lstinline[style=inline]!\doteq!       
    &$\checkmark$& \lstinline[style=inline]!\checkmark! 
    &   \\   
\end{tabular}
\end{center}

\section{Operadores de tamaño variable}
La suma y la integral se expanden cuando están en formato desplegado.
\begin{center}
\begin{tabular}{@{}*{3}{l@{\hspace{1.2em}}}@{}}
    $\sum_{k=0}^3k^2$ & $\dsum_{k=0}^3k^2$ & \lstinline[style=inline]!\sum_{k=0}^3 k^2! \\
    $\int_{0}^3x^2\,dx$ & $\dint_{0}^3x^2\,dx$ & \lstinline[style=inline]!\int_{x=0}^3 x^2\,dx!
\end{tabular}
\end{center}
Lo mismo ocurre con
\begin{center}
  \begin{tabular}{@{}*{6}{l@{\hspace{1.2em}}}@{}}
    $\int$&\lstinline[style=inline]!\int! 
    &$\iiint$&\lstinline[style=inline]!\iiint!
    &$\bigcup$&\lstinline[style=inline]!\bigcup! \\ 
    $\iint$&\lstinline[style=inline]!\iint!
    &$\oint$&\lstinline[style=inline]!\oint!       
    &$\bigcap$&\lstinline[style=inline]!\bigcap! 
\end{tabular}
\end{center}

\section{Flechas} 
\begin{center}
\begin{tabular}{@{}*{4}{l@{\hspace{1.5em}}}@{}}
    $\rightarrow$& \lstinline[style=inline]!\rightarrow!,\lstinline[style=inline]!\to!
    &$\mapsto$& \lstinline[style=inline]!\mapsto!   \\  
    $\nrightarrow$& \lstinline[style=inline]!\nrightarrow!     
    &$\longmapsto$& \lstinline[style=inline]!\longmapsto!\\
    $\longrightarrow$& \lstinline[style=inline]!\longrightarrow!
    &$\leftarrow$& \lstinline[style=inline]!\leftarrow!\\
    $\Rightarrow$& \lstinline[style=inline]!\Rightarrow!
    &$\leftrightarrow$& \lstinline[style=inline]!\leftrightarrow! \\
    $\nRightarrow$& \lstinline[style=inline]!\nRightarrow!
    &$\downarrow$& \lstinline[style=inline]!\downarrow!  \\   
    $\Longrightarrow$& \lstinline[style=inline]!\Longrightarrow!
    &$\uparrow$& \lstinline[style=inline]!\uparrow!  \\
    $\leadsto$& \lstinline[style=inline]!\leadsto!
    &$\updownarrow$& \lstinline[style=inline]!\updownarrow!  \\
  \end{tabular}
\end{center}


\section{Delimitadores}
\begin{center}
\begin{tabular}{@{}*{6}{l@{\hspace{1.5em}}}@{}}
    $(\;)$& \lstinline[style=inline]!()!
    &$\langle\;\rangle$&\lstinline[style=inline]!\langle\rangle!  
    &$|\; |$&\lstinline[style=inline]!| |!\\
    $[\;]$&\lstinline[style=inline]![]!       
    &$\lfloor\;\rfloor$&\lstinline[style=inline]!\lfloor\rfloor!
    &$\|\;\|$&\lstinline[style=inline]!\| \|!  \\
    $\{\;\}$&\lstinline[style=inline]!\{\}! 
    &$\lceil\;\rceil$&\lstinline[style=inline]!\lceil\rceil!  
\end{tabular}
\end{center}
Para ajustar al tamaño de la fórmula delimitada utilizar \lstinline[style=inline]!\left. \right.!.
\begin{center}
  $\displaystyle\left( i,2^{2^i}\right)$\quad
    \lstinline[style=inline]!\left( i,2^{2^i}\right)!
\end{center}
Cada
\lstinline[style=inline]!\left.! debe conicidir con
\lstinline[style=inline]!\right.! Para delimitadores de un solo lado colocar un punto.
\lstinline[style=inline]!\left.! o
\lstinline[style=inline]!\right.!.
\begin{center}
  $\displaystyle\left.\frac{df}{dx}\right|_{x_0}$\quad
  \lstinline[style=inline]!\left.\frac{df}{dx}\right|_{x_0}!
\end{center}
Se puede ajustar el tamaño con 
\lstinline[style=inline]!\big!, 
\lstinline[style=inline]!\Big!, 
\lstinline[style=inline]!\bigg!, or 
\lstinline[style=inline]!\Bigg!.
\begin{center}
  $\displaystyle\Big[\sum_{k=0}^n e^{k^2}\Big]$\quad
    \lstinline[style=inline]!\Big[\sum_{k=0}^n e^{k^2}\Big]!
\end{center}


\section{Arreglos, Matrices}
Las definiciones por casos es un arreglo de dos columnas.
\begin{center}
  $\displaystyle
  f_n=\begin{cases}
    a     &\text{si } n=0,   \\
    r\cdot f_{n-1}  &\text{caso contrario}.   
    \end{cases}
  $
\quad
\begin{minipage}{.7\columnwidth}\noindent
\begin{lstlisting}[xleftmargin=-1em]
  f_n=
  \begin{cases}
    a              &\text{si } n=0, \\
    r\cdot f_{n-1} &\text{caso contrario}.
  \end{cases}
\end{lstlisting}
\end{minipage}
\end{center}
Una matriz es otro arreglo que no necesita especificar las columnas.
\begin{center}
  $\displaystyle
  \begin{pmatrix}
    a  &b \\
    c  &d
  \end{pmatrix}
  $
  \quad
\begin{minipage}{.525\columnwidth}\noindent
\begin{lstlisting}[xleftmargin=0em]
  \begin{pmatrix}
    a  &b \\
    c  &d
  \end{pmatrix}
\end{lstlisting}
\end{minipage}
\end{center}
Se puede utilizar \lstinline[style=inline]!vmatrix! o \lstinline[style=inline]!bmatrix!.

\section{Ecuaciones desplegadas}
Coloca ecuaciones en una línea separada con el ambiente
\lstinline[style=inline]!equation*! o con \lstinline[style=inline]!\[ . \]!.
\begin{center}
\vspace*{-\topsep}
    \begin{minipage}{0.2\linewidth}\vspace*{-\abovedisplayskip}
        \begin{equation*}
          a^2+b^2=c^2
        \end{equation*}
    \end{minipage}
    \quad
    \begin{minipage}{0.3\linewidth}
        \begin{lstlisting}[xleftmargin=0ex]
\begin{equation*}
    a^2+b^2=c^2
\end{equation*}
        \end{lstlisting}
    \end{minipage}
\quad
    \begin{minipage}{0.3\linewidth}
        \begin{lstlisting}[xleftmargin=0ex]
\[
    a^2+b^2=c^2
\]
        \end{lstlisting}
    \end{minipage}
\end{center}
Puede escribir varias lineas sin alinear:
\begin{center}
\vspace*{-\topsep}
  \begin{minipage}{0.35\linewidth}\vspace*{-\abovedisplayskip}
    \begin{gather*}
      \sen (x)=x-\frac{x^3}{3!} \\
 +\frac{x^5}{5!}-\cdots
    \end{gather*}
  \end{minipage}
  \quad
  \begin{minipage}{0.55\linewidth}
\begin{lstlisting}[xleftmargin=0ex]
\begin{gather*}
  \sen (x)=x-\frac{x^3}{3!} \\
      +\frac{x^5}{5!}-\cdots
\end{gather*}
\end{lstlisting}
  \end{minipage}
\end{center}
Para alinear, usar el ambiente \lstinline[style=inline]!align*!.
\vspace*{-\topsep}
\begin{center}
  \begin{minipage}[c]{0.25\linewidth}\vspace*{-\abovedisplayskip}
    \begin{align*}
      f'(x) &= (x^2)' \\
            &= 2x    
    \end{align*}
  \end{minipage}
  \quad
  \begin{minipage}[c]{0.35\linewidth}
\begin{lstlisting}[xleftmargin=0ex,boxpos=c]
\begin{align*}
  f'(x) &= (x^2)' \\
        &= 2x    
\end{align*}
\end{lstlisting}
  \end{minipage}
\end{center}




\section{Leer más} 
Comprehensive \LaTeX{} Symbols 
List en \url{mirror.ctan.org/info/symbols/comprehensive}
y Math mode en \url{http://tex.loria.fr/general/Voss-Mathmode.pdf}.


\end{multicols}
\end{document}
